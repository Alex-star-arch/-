\documentclass{article}
\usepackage[UTF8]{ctex}
\usepackage{geometry}
\usepackage{multirow}
\usepackage{natbib}
\geometry{left=3.18cm,right=3.18cm,top=2.54cm,bottom=2.54cm}
\usepackage{graphicx}
\pagestyle{plain}	
\usepackage{setspace}
\usepackage{enumerate}
\usepackage{caption2}
\usepackage{datetime} %日期
\renewcommand{\today}{\number\year 年 \number\month 月 \number\day 日}
\renewcommand{\captionlabelfont}{\small}
\renewcommand{\captionfont}{\small}
\begin{document}

\begin{figure}
    \centering
    \includegraphics[width=8cm]{upc.png}

    \label{figupc}
\end{figure}

	\begin{center}
		\quad \\
		\quad \\
		\heiti \fontsize{45}{17} \quad \quad \quad 
		\vskip 1.5cm
		\heiti \zihao{2} 《计算科学导论》个人职业规划
	\end{center}
	\vskip 2.0cm
		
	\begin{quotation}
% 	\begin{center}
		\doublespacing
		
        \zihao{4}\par\setlength\parindent{7em}
		\quad 

		学生姓名:\underline{\qquad  万正浩 \qquad \qquad}

		学\hspace{0.61cm} 号:\underline{\qquad 1907030128\qquad}
		
		专业班级:\underline{\qquad 本研一体班(人工智能类)1901 \qquad  }
		
        学\hspace{0.61cm} 院:\underline{计算机科学与技术学院}
% 	\end{center}
		\vskip 1.5cm
		\centering
		\begin{table}[h]
            \centering 
            \zihao{4}
            \begin{tabular}{|c|c|c|c|c|c|c|c|c|}
            % 这里的rl 与表格对应可以看到,姓名是r,右对齐的;学号是l,左对齐的;若想居中,使用c关键字。
                \hline
                \multicolumn{5}{|c|}{分项评价} &\multicolumn{2}{c|}{整体评价}  & 总    分 & 评 阅 教 师\\
                \hline
                自我 & 环境 & 职业 & 实施 & 评估与 & 完整性 & 可行性 &\multirow{2}*{} &\multirow{2}*{}\\
                分析& 分析& 定位 & 方案 & 调整 & 20\% & 20\% & ~&~ \\\            
                10\% & 10\% & 15\% & 15\% & 10\% & &  &~ &~\\
                \cline{1-7} 
                & & & & & & & ~&~ \\
                & & & & & & & ~&~ \\
                \hline      
            \end{tabular}
        \end{table}
		\vskip 2cm
		\today
	\end{quotation}

\thispagestyle{empty}
\newpage
\setcounter{page}{1}
% 在这之前是封面,在这之后是正文
\section{自我分析}
	自我分析即对自己进行全方位、多角度的分析,目的是认识自己、了解自己。只有认识了自己,才能对自己的职业做出正确的选择,才能选定适合自己发展的职业生涯路线,才能对自己的职业生涯目标做出最佳抉择。\par
	自我分析包括:\par
\subsection{自然条件}
\begin{table}[h]
	\centering
	\caption{自然条件}
	\begin{tabular}{rl}
		\hline
		性别 & 男 \\
		年龄 & 17 \\ 
		身体条件 & 良好 \\
		健康状况 & 健康\\
		居住城市 & 山东聊城\\
		\hline
	\end{tabular}
	\label{table1}
\end{table}
\subsection{性格分析}
在生活中中比牧低调随和,不张扬;在人际交往中很容易相处,没有攻击性,且富有同情心,是一个很好的聆听者;说话比较委婉,能够敏感地觉察他人的情绪感受;在工作中你很稳重,喜欢按部就班的工作,同时也期待变化和新挑战。同时在工作生活中,目标感不强,主见弱,不擅于做决定;不喜欢崭露头角;当需要做出改变时往往犹豫不决;非常理性,对任何困难都进行理性分析。
\par
\subsection{教育与学习经历}
市重点初中,市重点高中,中国石油大学
\par
\subsection{工作与社会阅历}
多次社会实践活动,帮亲属在蛋糕房工作,自己与同学合伙创办辅导班并参与授课
\par
\subsection{知识、技能与经验}
团队中往往提出有力观点,具备领导能力;参加多次创新能力大赛,具备优秀的创新能力,多次为别人心理辅导,具有良好的逻辑分析能力和交际能力
\par
\subsection{兴趣爱好与特长}
音乐,自由拼装,收藏,滑雪,乒乓球等
\par
\section{环境分析}
环境分析主要是评估周边各种环境因素对自己职业生涯发展的影响。每一个人都处在一定的环境之中,职业发展必然要受到所处环境的影响,只有充分了解和把握所处环境的现状、特点、发展变化趋势,才能做到在复杂的环境中避害趋利,使你的职业生涯规划具有实际意义。\par
环境分析包括:\par
\subsection{社会环境分析}
中国现在正处于近两百年以来最好的历史时期。虽然社会上还有许多的体制存在弊端,还有许多没有解决的矛盾,但是政治上相对较稳定,法制化进程已经开始,市场经济已经初步形成并步入正轨。二十一世纪的中华大地需求较大,充满各种人才发展的机遇。\par
总体来说,我们面临一个非常好的宏观环境,社会安定,政治稳定,经济发展迅速,并与全球一体化接轨,法制建设不断完善,文化繁荣自由,尖端技术、高新技术突飞猛进。因此,在这个大前提之下,我们需要特别注意的是职业环境的变化。\par
\subsection{家庭环境分析}
婚姻状况:未婚\par
经济状况:无固定收入\par
家人期望:健康幸福\par
家族传统:无\par
\subsection{职业环境分析}
在政府及有关部门采取的一系列政策驱动下,如今的就业已不再难,如此,学生个人能力更应该成为当下一个值得关注的热点问题。如果我们把高校毕业生看作人才市场上的商品,那么各高校就是这些商品的生产企业。商品能不能满足用户要求,企业肯定是责无旁贷的。这也反映在专业课程的设置上,它反映的是高校教育的发展思路和教学资源的分配格局,它直接决定着大学生的知识结构和专业技能的掌握情况,并最终影响大学生的竞争力。\par

一般认为,当前大学生在学习和就业过程中会遇到一定的问题,其一是知识掌握不足或者学非所用,其二是实践能力有限。其实还有“其三”,那就是当前大学生对于求学及就业普遍显得过于功利,这多少偏离了教育的真正意义——不仅要增进一个人的知识与技能,更要培养其健全的人格,比如贡献社会、关怀弱势。相对于其一、其二,其三尤其重要,因为培养健全人格的过程就是认识社会、适应社会的过程,就是训练人与人关系处理技巧的过程,就是提高实践能力的过程,甚至是端正就业观念的过程。为此,高校在做好课堂教学的同时,还要盘活、用足校内外教育资源,通过各种校园文化活动、社会实践活动,促进学生能力发展,更好地促进他们人格的塑造。\par

眼下,有不少学校正集中精力为学生提供就业信息,但不要忘记,在这一过程中,还要对学生进行择业指导,要让学生有良好的职业规划意识和进行自我规划的能力。职业规划课需要未雨绸缪,从学生进大学时就开设,也就是说,高校要有解决大学生就业的长效机制。\par

大学毕业生就业,不仅仅是毕业季的事,前可追溯到入校之后就开始的职业生涯规划指导,后可延续到大学毕业生毕业离校后三五年甚至更长时间。从这一意义上说,解决大学毕业生就业,功夫在就业季之外,如果整个社会倡导公平就业的理念,大学高度重视人才培养质量,每个大学毕业生有清晰的职业规划,大学生的优势才会更加凸显,他们的就业之路会变得更加宽广顺畅。\par


\subsection{地域与人际环境分析}
首先,在中国这么一个人口众多的国家,尤其是在北京、上海这种一线城市,脱颖而出是每个人追求的目标,本科学历,四级证书已经成了最基本的标志。
努力学习工作几年后,薪水大概可以由3500上升到13000,远远高于在公司初期时的薪资水平,因为拥有一个好学历,并且多学习英语,就可以有更多的机遇。\par
其次,我们应该确立一个良好的职业发展方向,很多人,包括应届毕业生和工作了两三年的工作者,甚至有的人都工作了快5年的时间,仍然拿很低的薪水,勉强维持生计,感觉自己很迷茫,不知道能做什么,也不知道该做什么,这里,我提醒大家,IT已经不是曾经的泡沫经济时代了,如果你不喜欢从业这行,那么在你还没有进入这行之前,请三思。如果你已经选择了IT这个行业,那么恭喜你,虽然这个行业现在人数众多,但是百分之九十还都停留在最初级的IT民工层次。
\par 

\section{职业定位}
在准确地对自己和环境做出了分析之后,确定适合自己行业和有实现可能的职业发展目标。职业定位时要注意与自己的自然条件、知识背景、技能特长、性格特点、兴趣爱好是否匹配,考虑与自己所处的环境是否相适应。职业定位决定了职业发展中的行为和结果,是制定职业生涯规划的关键,应当科学合理,具有可行性。\par
职业定位包括:\par

\subsection{行业领域定位与理由}
◎INTJ详细描述
INTJ型的人是完美主义者。他们强烈地要求个人自由和能力,同时在他们独创的思想中,不可动摇的信仰促使他们达到目标。INTJ型的人 思维严谨、有逻辑性、足智多谋,他们能够看到新计划实行后的结果。他们对自己和别人都很苛求,往往几乎同样强硬地逼迫别人和自己。他们并不十分受冷漠与批评的干扰,作为所有性格类型中最独立的,INTJ型的人更喜欢以自己的方式行事。面对相反意见,他们通常持怀疑态度,十分坚定和坚决。权威本身不能强制地们,只有他们认为这些规则对自己的更重要的目标有用时,才会去遵守。INTJ型的人是天生的谋略家,具有独特的思想、伟大的远见和梦想。他们天生精于理论,对于复杂而综合的概念运转灵活。他们是优秀的战略思想家,通常能清楚地看到任何局势的利处和缺陷。对于感兴趣的问题,他们是出色的、具有远见和见解的组织者。如果是他们自己形成的看法和计划,他们会投入不可思议的注意力、能量和积极性。领先到达或超过自己的高标准的决心和坚忍不拔,使他们获得许多成就。\par
◎INTJ适合领域
科研、科技应用技术咨询、管理咨询金融、投资领域创造性行业\par
\par
\subsection{职业岗位起点定位与理由}
◎INTJ适合职业
各类科学家、研究所研究人员、设计工程师、系统分析员、计算机程序师、研究开发部经理等各类技术顾问、技术专家、企业管理顾问、投资专家、法律顾问、医学专家、精神分析学家等经济学家、投资银行研究员、证券投资和金融分析员、投资银行家、财务计划人、企业并购专家等各类发明家、建筑师、社论作家、设计师、艺术家等。
\par
\subsection{职业目标与可行性分析}
\par
成果目标、经济目标、能力目标、职务目标等。\par 
\begin{enumerate}[(1)]
	\item 短期目标(大学4年)\par 
	尽力进行各种职业的尝试,在学习中掌握更多的知识,在实践中提高能力 \par 
	\item 中长期目标(5-10年)\par 
	收入水平达到同行业人群的中高层,在同龄人群中达较高水平\par 
	
\end{enumerate}


\section{实施方案}
在明确了职业定位后,要制定实现职业生涯目标的行动方案,不付诸行动,职业目标只能是一种梦想。实施方案是实现职业目标的保证,尽量考虑周全、具有可操作性。\par
实施方案可以从以下角度考虑:\par
\begin{enumerate}[1、]
	\item 如何利用现有条件和自身优势以实现职业生涯目标。\par
	利用本研一体化培养中优秀的师资力量和环境积极提升自己,利用学校中的良好的学习条件进行实践,积极与导师取得联系,争取早一步进入导师的课题小组,进入实验室,发挥自身喜欢安静的特性,将自己的积极性投入到学习生活中,时刻告诉自己要积极参加课外活动,力争在活动中发挥领导作用,积极提升自己的各方面能力。\par
	\item 如何克服缺点、弥补不足、增长知识、提高能力以实现职业生涯目标。\par
	学习控制自己情绪的方法,做自己情绪的主人,积极调动自己的积极情绪进行学习活动,让学习和各种活动充实自己的生活,同时要注意与集体的联系,积极加入一些集体活动并发挥积极作用。\par
	\item 如何处理人际关系和发展人脉以实现职业生涯目标。\par
	在与人的交往方面,避免自己的一些牢骚等负能量,在与他人和群体的交流过程中积极传播正能量,在人脉方面,多结识一些有能力的人,可以在某方面的能力很突出,以取长补短,达到各方面能力提升的效果,与性格相似人交朋友,在对方的言行举止中看到自己,提升自己。\par
	\item 如何处理工作与家庭、生活的关系以实现职业生涯目标。
	有一个稳定的工作是组建一个美好家庭的前提,同时也是过一个幸福生活的前提,在这之间,应该建立一种理解,人与人之间,同时也要理解自己的工作和生活。\par
	\item 如何处理释放工作压力、保证身心健康以实现职业生涯目标。\par
	做情绪的主人,通过与亲近的朋友之间的交流,以及进行锻炼等方式倾述或者发泄自己的情感,同时也可以运用注意力转移法。\par
	
\end{enumerate}
\par 
\section{评估与调整}
由于影响职业生涯规划的因素很多,且大都处于动态变化之中,因此职业生涯规划应定期评估,并根据影响因素的变化和实施结果的情况及时作出调整,这样才能保证其行之有效。\par 
\subsection{评估时间}
每学期评估一次\par
\subsection{评估内容}
成果目标、能力目标、学习目标\par
确定哪些目标已按预期实现,哪些目标商未达到,对已实现的成果总结经验,对未完成的目标分析原因。\par
\subsection{调整原则}
根据目标完成情况和环境生活的变化考虑与自身情况的匹配性、与环境的适应性、操作实施的可行性等进行对规划的更改。\par




\end{document}
